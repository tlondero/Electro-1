\documentclass[a4paper]{article}
\usepackage[utf8]{inputenc}
\usepackage[spanish, es-tabla]{babel}

\usepackage{amsmath}
\usepackage{amsfonts}
\usepackage{amssymb}

\usepackage{float}
\usepackage{graphicx}

\usepackage{caption}
\usepackage{subcaption}
\captionsetup{compatibility=false}

\usepackage{multirow}
\setlength{\doublerulesep}{\arrayrulewidth}

\newcommand{\quotes}[1]{``#1''}
\newcommand\underrel[2]{\mathrel{\mathop{#2}\limits_{#1}}}

\usepackage{array}
\newcolumntype{C}[1]{>{\centering\let\newline\\\arraybackslash\hspace{0pt}}m{#1}}

\usepackage[american]{circuitikz}
\usepackage{xcolor}
\usepackage{fancyhdr}

\newlength{\stockheight}
\usepackage{hyperref}

\hypersetup{
    colorlinks=true,
    linkcolor=blue,
    filecolor=magenta,      
    urlcolor=blue,
    citecolor=blue,    
}

\urlstyle{same}

\usepackage{units} 
\pagestyle{fancy}
\fancyhf{}
\lhead{22.11 Electrónica I}
\rhead{Mechoulam, Lambertucci, Rodriguez, Londero}
\rfoot{Página \thepage}



\begin{document}

%%%%%%%%%%%%%%%%%%%%%%%%%%%%%%%%%%%%%%%%%%%%%%%%%%%%%%%%%%%%%%%%%%%%%%%%% 
%								CARATULA								%
%%%%%%%%%%%%%%%%%%%%%%%%%%%%%%%%%%%%%%%%%%%%%%%%%%%%%%%%%%%%%%%%%%%%%%%%% 

\begin{titlepage}
\newcommand{\HRule}{\rule{\linewidth}{0.5mm}}
\center
\mbox{\textsc{\LARGE \bfseries {Instituto Tecnológico de Buenos Aires}}}\\[1.5cm]
\textsc{\Large 22.11 Electrónica I}\\[0.5cm]


\HRule \\[0.6cm]
{ \Huge \bfseries Trabajo práctico N$^{\circ}$1}\\[0.4cm] 
\HRule \\[1.5cm]


{\large

\emph{Grupo 3}\\
\vspace{3px}

\begin{tabular}{lr} 	
\textsc{Mechoulam}, Alan  &  58438\\
\textsc{Lambertucci}, Guido Enrique  & 58009 \\
\textsc{Rodriguez Turco}, Martín Sebastian  & 56629 \\
\textsc{Londero Bonaparte}, Tomás Guillermo  & 58150 \\
\end{tabular}

\vspace{20px}

\emph{Profesores}\\
Alcocer, Fernando\\
Oreglia, Eduardo Victor\\
Gardella, Pablo Jesús\\
\vspace{3px}
%\textsc{} \\	

\vspace{100px}

\begin{tabular}{ll}

Presentado: & 24/09/19\\

\end{tabular}

}

\vfill

\end{titlepage}


\tableofcontents
\newpage

\begin{center}
	\textcolor{red}{\textbf{VERIFICAR FECHA DE ENTREGA.}}
\end{center}

%%%%%%%%%%%%%%%%%%%%%%%%%%%%%%%%%%%%%%%%%%%%%%%%%%%%%%%%%%%%%%%%%%%%%%%%% 
%								INFORME									%
%%%%%%%%%%%%%%%%%%%%%%%%%%%%%%%%%%%%%%%%%%%%%%%%%%%%%%%%%%%%%%%%%%%%%%%%%

\section{Introducción}
La configuración a estudiar en este informe será la configuración de \textbf{Bootstrap}. El bootstraping es una técnica usada para aumentar al impedancia de entrada y bajar la de salida, utilizando un capacitor para realimentar el emisor del transistor con su respectiva base. El circuito propuesto es un colector común utilizando el capcitor $C_2$ para la realimentación:

\begin{figure}[H]
\begin{center}
\begin{circuitikz}[european voltages]
	\node [npn](Q){};
	\draw (Q.B) to[short, -*] ++(-3,0) node[](v1){};
	\draw (v1) to[C, l = $C_1$] ++(-1.5,0) to[R, l = $R_S$] ++(-1.5,0) to[short] ++(-0.5,0) node[ocirc,label=left:$V_{S}$]{};
	\draw (v1) to[short] ++(0,-0.5) to[R, l = $R_3$] ++(0,-1.5) to[short, -*] ++(1.5,0) node[](v2){} to[C, l = $C_2$] ++(2.35,0) to[short] ++(0.5,0) to[C, l = $C_3$] ++(1.5,0) node[](aux){};
	\draw (aux) to[open] ++(0,-2.1) node[ground]{} to[R, l= $R_L$, v= $V_o$] (aux);
	\draw (v2) to[short] ++(0,-0.5) to[R, l = $R_2$] ++(0,-1.5) node[ground]{};
	\draw (Q.E) to[short, -*] ++(0,-1.23) to[short] ++(0,-0.5) to[R, l = $R_e$] ++(0,-1.5) node[ground]{};
	\draw (Q.C) to[short] ++(0,2) node[ocirc,label=right:$V_{cc}$]{};
	\draw (v2) to[short] ++(0,2.5) to[R, l = $R_1$] ++(0,1.5) to[short, -*] ++(2.35,0);
\end{circuitikz}
\caption{Circuito Bootstrap propuesto.}
\label{fig:boot}
\end{center}
\end{figure}

\section{Desarrollo}

\subsection{Polarización}
El primer paso consiste en pasivar la fuente Vs y tratar los capacitores como un circuito abierto. Redibujando y utilizando el teorema de Thevenin se llega a las siguientes condiciones:
\begin{figure}[H]
\begin{center}
\begin{circuitikz}[european voltages]
	\node [npn](Q){};
	\draw (Q.B) to[R, l = $R_{Th}$] ++(-1.5,0) to[short] ++(-0.5,0) node[ocirc,label=left:$V_{Th}$]{};
	\draw (Q.E) to[R, l = $R_e$] ++(0,-1.5) node[ground]{};
	\draw (Q.C) to[short] ++(0,0.5) node[ocirc,label=right:$V_{cc}$]{};
\end{circuitikz}
	\caption{Circuito Bootstrap Polarización.}
	\label{fig:pol}
\end{center}
\end{figure}

donde:

\begin{equation*}
\left\{
\begin{aligned}
		& V_{Th}= V_{cc}\cdot \frac{R_2}{R_1+R_2} \\
		& R_{Th}= (R_1 // R_2) + R_3 
\end{aligned}
\right.
\end{equation*}

Recorriendo la malla de entrada se obtienen la siguientes ecuaciones:
\begin{equation*}
	V_{Th}-I_b \cdot R_{Th} -V_{BE_{On}}-I_e \cdot R_E=0  \ \  I_b\approx  \beta \cdot I_e
\end{equation*}

Es así que se consigue una expresión para $I_C$ con $V_{ce}$:
\begin{equation*}
	I_{C}\approx \frac{V_{Th}-V_{BE_{On}}}{\frac{R_{Th}}{1+\beta}+R_E} \ \ V_{ce} = V_{cc}-I_c\cdot R_E
\end{equation*}

\subsection{Modelo Incremental}
Para el modelo incremental se utilizarán los siguientes estimadores:
\begin{equation*}
\left\{
\begin{aligned}
	& \hat{hfe}=\beta \\
	& \hat{hie} = \frac{V_T}{I_b} \\
	& \hat{\frac{1}{hoe}} = \frac{V_a}{I_c}
\end{aligned}
\right.
\end{equation*}

Siendo este el circuito correspondiente al modelo:
\begin{figure}[H]
\begin{center}
\begin{circuitikz}[european voltages]
	\node [ocirc,label=left:$B$](b){};
	\draw (b) to[short, f_= $I_B$] ++(1.5,0) to[R, l = $hie$] ++(0,-2) node[circ](e){} to[short] ++(-1.5,0) node[ocirc, label=left:$E$](){};
	\draw (e) to[short, -*] ++(2,0) to[open] ++(0,2) node[](aux){} to[dcisource, l = $hfe \cdot I_B$] ++(0,-2) to[short, -*] ++(2.5,0) to[open] ++(0,2) to[R, l= $\frac{1}{hoe}$] ++(0,-2) to[short]++(1.5,0) node[ocirc, label=right:$E$](){};
	\draw (aux) to[short] ++(4,0) node[ocirc, label=right:$C$](){};
\end{circuitikz}
	\caption{Modelo incremental.}
	\label{fig:modinc}
\end{center}
\end{figure}

Para el transistor utilizado en este informe, se midió HFE, siendo este $HFE \approx 380$
\subsection{Circuito Incremental}
Reemplazando el transistor por su modelo incremental, asumiendo que se trabaja con pequeñas señales a frecuencias medias, se obtiene el siguiente circuito:
\begin{figure}[H]
\begin{center}
\begin{circuitikz}[european voltages]
	\node [ocirc,label=left:$V_i$](vi){};
	\draw (vi) to[short] ++(0.5,0) to[R, l = $R_S$] ++(2,0) node[circ](v1){} to[R, l = $R_3$] ++(2,0) node[circ](v2){};
	\draw (v1) to[short] ++(0,1) to[R, l = $hie$] ++(2,0) to[short] ++(0,-1);
	\draw (v2) to[open] ++(0,-2) node[ground](gnd){} to[dcisource, l = $hfe \cdot I_B$] ++(0,2);
	\draw (v2) to[short, -*] ++(1.5,0) to[R, l= $\frac{1}{hoe}$] ++(0,-2) node[ground](){};
	
	\draw (v2) to[short, -*] ++(3,0) to[R, l= $R_1$] ++(0,-2) node[ground](){};
	\draw (v2) to[short, -*] ++(4.5,0) to[R, l= $R_2$] ++(0,-2) node[ground](){};
	\draw (v2) to[short, -*] ++(6,0) to[R, l= $R_E$] ++(0,-2) node[ground](){};
	\draw (v2) to[short, -*] ++(8,0) to[open] ++(0,-2) node[ground](){} to[R, l= $R_L$, v = $V_o$] ++(0,2);
\end{circuitikz}
	\caption{Circuito incremental.}
	\label{fig:circinc}
\end{center}
\end{figure}

Redibujando convenientemente y utilizado el pasaje a nivel de corriente, se puede describir el circuito como:
\begin{figure}[H]
\begin{center}
\begin{circuitikz}[european voltages]
	\node [ocirc,label=left:$V_i$](vi){};
	\draw (vi) to[short] ++(0.5,0) to[R, l = $R_S + \left( R_3 // hie \right)$] ++(2,0) to[open] ++(0,-2) node[ground](){} to[R, l= $R_{d}$, v = $V_o$] ++(0,2);
\end{circuitikz}
	\caption{Circuito incremental final.}
	\label{fig:circinc2}
\end{center}
\end{figure}
siendo
\begin{equation*}
	R_d = \frac{1}{hoe} // R_1 // R_2 // R_E // R_L \cdot \left(1 + \frac{R_3 hfe}{R_3 + hfe}\right) 
\end{equation*}

Una pequeña modificacion que se hace a la expresión del circuito es definir un $hfe$ y un $hie$ efectivo. Para el circuito dado, estos son:
\begin{equation*}
\left\{
\begin{aligned}
	& hfe^* = hfe\cdot \frac{r_3}{R_3+hie} \\
	& hie^* = R_3 // hie
\end{aligned}
\right.
\end{equation*}

\subsection{Elección de componentes.}

Se eligió un punto de operación \textbf{Q} tal que la tensión colector-emisor $V_{ce} = 5 V$,mientras que la corriente de colector $I_{cq} = 2 \ mA$.  Se eligieron resistencia que cumplan dicho punto de polarización, recorriedo la malla de entrada y salida para determinarlos. Es así que se llegó a los siguientes valores: $R_1 = 10 \ k\Omega$, $R_2 = 14.7 \ k\Omega$, $R_3 = 1 \ k\Omega$ y $R_E = 2.2 \ k\Omega$.

Por otro lado, se tomó como resistencia del generador $R_s \approx 50 \ \Omega$, mientras que para la carga se consideró $R_L \approx 10 \ k\Omega$. Para los capacitores, sus valores nominales fueron elegidos tal que su impedancia sea despreciable frente a las resistencias frente a frecuencias medias. Finalmente se eligió una tensión de alimentación vcc=10V.
\subsection{Resultados de interés.}
Se realiza un enfoque en la ganancia de tensión, corriente, impedancia de entrada y de salida:


\begin{equation}
	\Delta V \triangleq \frac{V_o}{V_i} = \frac{ \left(\frac{1}{hoe} // R_1 // R_2 // R_E // R_L  \right)\cdot (1+hfe^*)}{R_s + hie^* + \left(\frac{1}{hoe} // R_1 // R_2 // R_E // R_L  \right)\cdot (1+hfe*) } 
\end{equation}

\begin{equation} \Delta I \triangleq \frac{I_o}{I_i} =  \frac{\frac{1}{hoe} // R_1 // R_2 // R_E}{\frac{1}{hoe} // R_1 // R_2 // R_E+R_L} \cdot (1+hfe^*)
\end{equation}
\begin{equation} Z_{In} = R_s + hie^* + \left(\frac{1}{hoe} // R_1 // R_2 // R_E // R_L  \right)\cdot (1+hfe*)\end{equation}
\begin{equation} Z_{Out} = \frac{(R_s + hie^*)}{(1+hfe*)} //  \left(\frac{1}{hoe} // R_1 // R_2 // R_E // R_L  \right)\end{equation}

Luego se graficó la ganancia de tensión en función de la frecuencia:
\begin{figure} [H]
	\centering
	\includegraphics[width=\textwidth]{imagenes/avs.png}
	\caption{Transferencia Módulo.}
	\label{fig:transmod}
\end{figure}
\begin{figure} [H]
	\centering
	\includegraphics[width=\textwidth]{imagenes/avsp.png}
	\caption{Transferencia Fase.}
	\label{fig:transph}
\end{figure}
Se midió la impedancia de entrada para frecuencias medias, las cuales fueron entre 1kHz y 10 kHz, obteniendio un valor de $Z_{in}=83.4k\Omega$ y midiendo la impedancia de salida se obtuvo un valor de $Z_{out} = 9.86 k \Omega $
Dichas mediciones coinciden parcialmente con la simulación, las discrepancias se deben principalmente a los cambios en el HFE, los cuales son esperables, pero dadoa  que la impedancia de entrada es grande y la de salida pequeña se toman como válidos.

\end{document}
