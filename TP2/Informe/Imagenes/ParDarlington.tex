\documentclass[border={0.5cm 0.5cm 0.5cm 0.5cm}, 11pt, tikz, multi=page]{standalone}
\usepackage[utf8]{inputenc}
\usepackage[spanish, es-tabla, es-noshorthands]{babel}

\usepackage[a4paper, footnotesep = 1cm, width=18cm, left=2cm, top=2.5cm, height=25cm, textwidth=18cm, textheight=25cm]{geometry}
%\geometry{showframe}

\usepackage{tikz}
\usepackage{textcomp}
\usetikzlibrary{shapes,arrows}

\usepackage{amsmath}
\usepackage{amsfonts}
\usepackage{amssymb}
\usepackage{float}
\usepackage{graphicx}
\usepackage{caption}
\usepackage{subcaption}
\usepackage{multicol}
\usepackage{multirow}
\setlength{\doublerulesep}{\arrayrulewidth}
\usepackage{booktabs}

\usepackage{hyperref}
\hypersetup{
    colorlinks=true,
    linkcolor=blue,
    filecolor=magenta,      
    urlcolor=blue,
    citecolor=blue,    
}

\newcommand{\quotes}[1]{``#1''}
\usepackage{array}
\newcolumntype{C}[1]{>{\centering\let\newline\\\arraybackslash\hspace{0pt}}m{#1}}
\usepackage[american]{circuitikz}
\usepackage{fancyhdr}
\usepackage{units}

% Definition of blocks:
\tikzset{%
  block/.style    = {draw, thick, rectangle, minimum height = 3em,
    minimum width = 3em},
  sum/.style      = {draw, circle, node distance = 2cm}, % Adder
  input/.style    = {coordinate}, % Input
  output/.style   = {coordinate}, % Output
  >=Stealth
}

% Defining string as labels of certain blocks.
\newcommand{\suma}{\Large $\Sigma$}
\newcommand{\inte}{$\displaystyle \int$}
\newcommand{\derv}{\huge $\frac{d}{dt}$}

\begin{document}

%Darlington sin compensar
\begin{page}
\begin{circuitikz}
	\node [npn, label=right:$Q_1$](n1){};
	\draw (n1.E) |- ++(0.5,-0.5) node[npn, anchor=B, label=right:$Q_2$](n2){};
	\draw (n2.C) -- ++(0,2) node[](we){};
	\draw (we.center) -- ++(0,1) node[ocirc, label=above:$V_{CC}$](){};
	\draw (n2.E) to[R, l=$R_E$] ++(0,-1.5) node[ground](){};
	\draw (n1.B) -- ++(-1,0) to[R, l=$R_B$] ++(0,1.5) |- (we.center);
	\draw (n1.C) |- (we.center);
\end{circuitikz}
\end{page}

%Darlington compensado por R
\begin{page}
\begin{circuitikz}
	\node [npn, label=right:$Q_1$](n1){};
	\draw (n1.E) -- ++(0,-0.5) node[](aux1){};
	\draw (aux1.center) -- ++(2,0) node[npn, anchor=B, label=right:$Q_2$](n2){};
	\draw (n2.C) -- ++(0,2) node[](we){};
	\draw (we.center) -- ++(0,1) node[ocirc, label=above:$V_{CC}$](){};
	\draw (n2.E) -- ++(0,-0.5) node[](aux2){};
	\draw (aux2) to[R, l=$R_E$] ++(0,-2.5) node[ground](){};
	\draw (aux2.center) -- ++(-1,0) to[R] ++(-1.5,0) -| (aux1.center);
	\draw (n1.B) -- ++(-1,0) to[R, l=$R_B$] ++(0,1.5) |- (we.center);
	\draw (n1.C) |- (we.center);
\end{circuitikz}
\end{page}

%Darlington compensado por fuente de I
\begin{page}
\begin{circuitikz}
	\node [npn, label=right:$Q_1$](n1){};
	\draw (n1.E) -- ++(0,-0.5) node[](aux1){};
	\draw (aux1.center) -- ++(2,0) node[npn, anchor=B, label=right:$Q_2$](n2){};
	\draw (n2.C) -- ++(0,2) node[](we){};
	\draw (we.center) -- ++(0,1) node[ocirc, label=above:$V_{CC}$](){};
	\draw (n2.E) -- ++(0,-0.5) node[](aux2){};
	\draw (aux2) to[dcisource] ++(0,-1.5) node[ground](){};
	
	\draw (aux1.center) -- ++(0,-0.5) to[R, l=$R_E$] ++(0,-2.25) node[ground](){};
	%\draw (aux2.center) -- ++(-1,0) to[R, l=$R_E$] ++(-1.5,0) -| (aux1.center);
	\draw (n1.B) -- ++(-1,0) to[R, l=$R_B$] ++(0,1.5) |- (we.center);
	\draw (n1.C) |- (we.center);
\end{circuitikz}
\end{page}

%Fuente de I
\begin{page}
\begin{circuitikz}[european voltages]
	\node [njfet, label=right:$Q_3$](jf){};
	\draw (jf.S) to[R, l=$R_S$] ++(0,-2) node[](aux1){} -- ++(0,-0.5) node[label=right:-$V_{SS}$, ocirc](vss){};
	\draw (jf.D) to[R, f<_=$I_o$, l=$R_D$] ++(0,2) node[ocirc, label=right:$V_{DD}$](){};
	\draw (jf.G) -- ++(-1,0) -- ++(0,-2.5) node[ground](){};
\end{circuitikz}
\end{page}

%Darlington reemplazando I por Rof
\begin{page}
\begin{circuitikz}[european voltages]
	\node [npn, label=right:$Q_1$](n1){};
	\draw (n1.E) |- ++(0.5,-0.5) node[npn, anchor=B, label=right:$Q_2$](n2){};
	\draw (n2.C) -- ++(0,2) node[](we){};
	\draw (we.center) -- ++(0,1) node[ocirc, label=above:$V_{CC}$](){};
	\draw (n2.E) to[dcisource, l=$I_{DS}$] ++(0,-1.5) node[ground](){};
	\draw (n1.B) -- ++(-1,0) to[R, l_=$R_B$] ++(0,1.5) |- (we.center);
	\draw (n1.C) |- (we.center);
	\draw (n1.E) -- ++(0,-1.25) to[R, l_=$R_E$] ++(0,-1.5) node[ground](){};
	
	%\draw (n2.E) to[R, l=$R_E$] ++(-2.75,0) -| (n1.E);	
	
	\draw (n1.B) ++(-1,0) to[C, l=$C_1$] ++(-1.5,0) to[R, l=$R_S$] ++(-2,0) to[sV, l_=$V_S$] ++(0,-2) node[ground](){};
	\draw (n2.E) to[C, l=$C_2$] ++(2.5,0) to[R, l=$R_L$] ++(0,-1.5) node[ground](){};
\end{circuitikz}
\end{page}

\end{document}