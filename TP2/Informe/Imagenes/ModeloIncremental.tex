\documentclass[border={0.5cm 0.5cm 0.5cm 0.5cm}, 11pt, tikz, multi=page]{standalone}
\usepackage[utf8]{inputenc}
\usepackage[spanish, es-tabla, es-noshorthands]{babel}

\usepackage[a4paper, footnotesep = 1cm, width=18cm, left=2cm, top=2.5cm, height=25cm, textwidth=18cm, textheight=25cm]{geometry}
%\geometry{showframe}

\usepackage{tikz}
\usepackage{textcomp}
\usetikzlibrary{shapes,arrows}

\usepackage{amsmath}
\usepackage{amsfonts}
\usepackage{amssymb}
\usepackage{float}
\usepackage{graphicx}
\usepackage{caption}
\usepackage{subcaption}
\usepackage{multicol}
\usepackage{multirow}
\setlength{\doublerulesep}{\arrayrulewidth}
\usepackage{booktabs}

\usepackage{hyperref}
\hypersetup{
    colorlinks=true,
    linkcolor=blue,
    filecolor=magenta,      
    urlcolor=blue,
    citecolor=blue,    
}

\newcommand{\quotes}[1]{``#1''}
\usepackage{array}
\newcolumntype{C}[1]{>{\centering\let\newline\\\arraybackslash\hspace{0pt}}m{#1}}
\usepackage[american]{circuitikz}
\usepackage{fancyhdr}
\usepackage{units}

% Definition of blocks:
\tikzset{%
  block/.style    = {draw, thick, rectangle, minimum height = 3em,
    minimum width = 3em},
  sum/.style      = {draw, circle, node distance = 2cm}, % Adder
  input/.style    = {coordinate}, % Input
  output/.style   = {coordinate}, % Output
  >=Stealth
}

% Defining string as labels of certain blocks.
\newcommand{\suma}{\Large $\Sigma$}
\newcommand{\inte}{$\displaystyle \int$}
\newcommand{\derv}{\huge $\frac{d}{dt}$}

\begin{document}

%Modelo incremental fuente de I
\begin{page}
\begin{circuitikz}[european voltages]
	\node [ground, label=below left:$G$](g1){};
	\draw (g1) to[R, l=$R_{GS}$] ++(0,2) -- ++(1.5,0) node[circ,label=above left:$S$](s){} to[R, l_=$R_S$] ++(0,-2) node[ground](){};
	\draw (s) to[open] ++(2.5,0) node[circ,label=above right:$D$](d){} to[R, l=$R_D$, f<=$I_o$] ++(2.5,0) node[ocirc](vo){} ++(0,-2.5) node[](-vo){};
	\draw (d) to[dcisource, l_=$gm V_{GS}$] (s);
	\draw (-vo) to[open, v = $V_o$] (vo);
	\draw (s) -- ++(0,1.5) to[R, l=$R_{DS}$] ++(2.5,0) -- (d);
\end{circuitikz}
\end{page}

%Modelo incremental fuente de I simplificado
\begin{page}
\begin{circuitikz}[european voltages]
	\node [circ,label=above left:$S$](s){};
	\draw (s) to[R, l_=$R_{GS}^*$, f>=$I_o$] ++(0,-2) node[ground, label=below left:$G$](){} ++(-1,0) node[](aux1){};
	\draw (s) to[dcisource, l=$gm V_S$] ++(2.5,0) node[circ,label=above right:$D$](d){} to[R, l=$R_D$, f<=$I_o$] ++(2.5,0) node[ocirc](vo){} ++(0,-2.5) node[](-vo){};
	\draw (-vo) to[open, v = $V_o$] (vo);
	\draw (s) -- ++(0,1.5) to[R, label=$R_{DS}$, f<=$I_{DS}$] ++(2.5,0) -- (d);
	\draw (aux1) ++(0,-0.5) to[open, v^=$V_S$] ++(0,2.5);
\end{circuitikz}
\end{page}

%Modelo incremental Darlington
\begin{page}
\begin{circuitikz}[european voltages]
	\node [circ,label=above:$B_1$](b1){};
	\draw (b1) to[R, l=$R_B$] ++(0,-2) node[ground](){};
	\draw (b1) to[R, l=$h_{ie1}$, f=$I_{B1}$] ++(2,0) node[](aux1){} ++(0,-2) node[ground](aux2){};
	\draw (aux2) to[dcisource, l_=$h_{fe1} I_{B1}$] (aux1) -- ++(2.5,0) node[circ, label=above:$E_1 \equiv B_2$](aux3){} to[R, l=$\frac{1}{h_{oe1}}$] ++(0,-2) node[ground](){};
	\draw (aux3) -- ++(1.5,0) to[R, l=$R_E$] ++(0,-2) node[ground](){};
	\draw (aux3) ++(1.5,0) to[R, l=$h_{ie2}$, f=$I_{B2}$] ++(2,0) ++(0,-2) node[ground](){} to[dcisource, l_=$h_{fe2} I_{B2}$] ++(0,2) -- ++(2.5,0) node[circ, label=above:$E_2$](aux4){} to[R, l=$R_{OF}$] ++(0,-2) node[ground](){};
	\draw (aux4.center) -- ++(1.5,0) to[R, l=$\frac{1}{h_{oe2}}$] ++(0,-2) node[ground](){};
	
	\draw (b1) to[R, l_=$R_g$, f_<=$I_i$] ++(-2.5,0) to[sV, l_=$V_S$] ++(0,-2) node[ground, label=left:$C_1 \equiv C_2$](){};
	\draw (aux4.center) ++(1.5,0) to[short, f=$I_o$] ++(1.5,0) to[R, l=$R_L$] ++(0,-2) node[ground](){} ++(1,0) node[](-vo){};
	\draw (-vo) to[open, v = $V_o$] ++(0,2);
	\draw (b1) ++(-0.25,-2) to[open, v^= $V_i$] ++(0,2);
\end{circuitikz}
\end{page}

%Modelo incremental Darlington con paralelos 1
\begin{page}
\begin{circuitikz}[european voltages]
	\node [circ,label=above:$B_1$](b1){};
	\draw (b1) to[R, l=$R_B$] ++(0,-2) node[ground](){};
	\draw (b1) to[R, l=$h_{ie1}$, f=$I_{B1}$] ++(2,0) node[](aux1){} ++(0,-2) node[ground](aux2){};
	\draw (aux2) to[dcisource, l_=$h_{fe1} I_{B1}$] (aux1) to[short] ++(2.5,0) node[circ, label=above:$E_1 \equiv B_2$](aux3){};
	\draw (aux3) to[R, l=$R_{O1}^{*}$] ++(0,-2) node[ground](){};
	\draw (aux3) to[R, l=$h_{ie2}$, f=$I_{B2}$] ++(3,0) ++(0,-2) node[ground](){} to[dcisource, l_=$h_{fe2} I_{B2}$] ++(0,2) to[short] ++(2.5,0) node[circ, label=above:$E_2$](aux4){} to[R, l=$R_d$] ++(0,-2) node[ground](){} ++(1,0) node[](-vo){};
	\draw (-vo) to[open, v = $V_o$] ++(0,2);
	\draw (b1) to[R, l_=$R_g$, f_<=$I_i$] ++(-2.5,0) to[sV, l_=$V_S$] ++(0,-2) node[ground, label=left:$C_1 \equiv C_2$](){};
	\draw (b1) ++(-0.25,-2) to[open, v^= $V_i$] ++(0,2);
\end{circuitikz}
\end{page}

%Modelo incremental Darlington con paralelos 2
\begin{page}
\begin{circuitikz}[european voltages]
	\node [circ,label=above:$B_1$](b1){};
	\draw (b1) to[R, l=$R_B$] ++(0,-2) node[ground](){};
	\draw (b1) to[R, l=$h_{ie1}$, f=$I_{B1}$] ++(2,0) node[](aux1){} ++(0,-2) node[ground](aux2){};
	\draw (aux2) to[dcisource, l_=$h_{fe1} I_{B1}$] (aux1) to[short] ++(2.5,0) node[circ, label=above:$E_1 \equiv B_2$](aux3){};
	\draw (aux3) to[R, l=$R_{O1}^{*}$] ++(0,-2) node[ground](){};
	\draw (aux3) to[R, l=$h_{ie2}$, f=$I_{B2}$] ++(3,0) node[circ, label=above right:$E_2$](aux4){} to[R, l=$R_d \left(1 + h_{fe2} \right)$] ++(0,-2) node[ground](){};
	\draw (b1) to[R, l_=$R_g$, f_<=$I_i$] ++(-2,0) to[sV, l_=$V_S$] ++(0,-2) node[ground, label=left:$C_1 \equiv C_2$](){};
	\draw (b1) ++(-0.25,-2) to[open, v^= $V_i$] ++(0,2);
\end{circuitikz}
\end{page}

%Modelo incremental Darlington con paralelos 3
\begin{page}
\begin{circuitikz}[european voltages]
	\node [circ,label=above:$B_1$](b1){};
	\draw (b1) to[R, l=$R_B$] ++(0,-2) node[ground](){};
	\draw (b1) to[R, l=$h_{ie1}$, f=$I_{B1}$] ++(2,0) node[](aux3){};
	\draw (aux3) to[R, l=$R_{d}^* \left( 1 + h_{fe1} \right)$] ++(0,-2) node[ground](){};
	\draw (b1) to[R, l_=$R_g$, f_<=$I_i$] ++(-3,0) to[sV, l_=$V_S$] ++(0,-2) node[ground, label=left:$C_1 \equiv C_2$](){};
	\draw (b1) ++(-0.25,-2) to[open, v^= $V_i$] ++(0,2);
\end{circuitikz}
\end{page}

%Modelo incremental Darlington Roa
\begin{page}
\begin{circuitikz}[european voltages]
	\node [circ,label=above:$B_1$](b1){};
	\draw (b1) to[R, l_=$R_B // R_g$] ++(0,-2) node[ground, label=left:$C_1 \equiv C_2$ ](){};
	\draw (b1) to[R, l=$h_{ie1}$, f=$I_{B1}$] ++(2,0) node[](aux1){} ++(0,-2) node[ground](aux2){};
	\draw (aux2) to[dcisource, l_=$h_{fe1} I_{B1}$] (aux1) -- ++(2.5,0) node[circ, label=above:$E_1 \equiv B_2$](aux3){} to[R, l=$R_{O1}^{*}$, f_<=$I_{E}^{*}$] ++(0,-2) node[ground](){};
	\draw (aux3)  to[R, l=$h_{ie2}$, f=$I_{B2}$] ++(3,0) ++(0,-2) node[ground](){} to[dcisource, l_=$h_{fe2} I_{B2}$] ++(0,2) -- ++(2.5,0) node[circ, label=above:$E_2$](aux4){} to[R, l=$R_{OF} // \frac{1}{h_{oe2}}$] ++(0,-2) node[ground](){};
	\draw (aux4.center) to[short, f<=$I_o$] ++(1.5,0) node[ocirc](+vo){};
	\draw (+vo) ++(0.5,0) to[open, v^<= $V_o$] ++(0,-2);
\end{circuitikz}
\end{page}


\end{document}